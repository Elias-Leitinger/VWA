\chapter{Historical Generational Developments of Nuclear Reactors}
\myfig{timeline}
    {height=0.25\textheight}
    {Timeline of nuclear reactor generations}
    {Timeline of nuclear reactor generations}
    {fig:timeline}
Nuclear reactors are normally categorised into generations. These generations differentiate
reactors based on various technological factors such as cost-effectiveness, safety, commercial applicability and
fuel cycle. They present a useful tool in categorizing nuclear reactors as multiple factors
are combined into one single metric. It is important to note that all designs of generations I
to III+ consist mainly of variations on designs first developed in the late fourth decade of the 20th century with
various incremental improvements added. \footcite[1,2]{Gen2gen}.
Figure~\ref{fig:timeline} shows a timeline of nuclear reactor generations along with important notes
on the properties of reactors in this generation. The images show representatives of each generation.
It is important to remember that the example image for generation IV is not representative, as the reactors
of this generation are still in development and have not been built to commercial scale.
\section{Generation I}
The first generation of nuclear reactors consists primarily of research reactors and primitive nuclear
power plants. These reactors are regarded as \enquote{proof of concept} in the USA. All reactors of
this generation have been decommissioned or are undergoing deconstruction because their technological level
is far behind that of newer reactors. Therefore they have very low cost-effectiveness and operational
safety \footcite[3]{Gen2gen}.
\section{Generation II}
The second generation of nuclear reactors represent the first efforts to produce nuclear reactors primarily
designed for commercial viability. They comprise mainly boiling water reactors (BWR), pressurised water reactors (PWR),
Canadian deuterium uranium reactors (CANDU) and advanced gas-cooled reactors (AGR).  They are designed
for an operational lifetime of 40 years. Generation II reactors are the most common generation for boiling
water reactors and pressurised water reactors around the world, which can be categorised under the term
light water reactor (LWR). These reactors feature more advanced safety features compared to generation I
reactors. These safety features have the ability to automatically prevent grievous incidents in the operation
of the nuclear reactor, such loss of power or operator error.

Designs from this category require comparatively large electrical power grids. And feature advanced
safety envelopes based on western standards. New reactors of this generation are mainly built in
China, Russia and the Republic of Korea. Their fuel cycle requires high level waste repositories
for ultimate disposition \footcite[4-6]{Gen2gen}.
\section{Generation III}
In essence nuclear reactors of the third generation are the same designs as in the second generation
with evolutionary improvements. These reactors feature improved fuel economy, thermal efficiency and
and safety. These designs shift away from active safety systems to passive safety systems. They also
feature a more standardised design, which leads to more economical construction costs. Designs such
as the advanced boiling water reactor (ABWR) belong to this generation \footcite[5,6]{Gen2gen}.
\section{Generation III+}
As implied by the name nuclear reactors of the III+ generation feature another set of incremental improvements over the third
generation. The main focus of these improvements lay in the improved safety of reactor systems as
less operator intervention and fewer active components are utilized in reactor safety systems. These
new safety systems utilize effects such as gravity to function. This leads to improved safety as a
total failure of these safety systems is very unlikely compared to those of earlier generations.
These reactors also have a higher expected operating lifetime of up to 60 years. The power plants of
this generation also have increased fuel burn up. This has the benefit of reduced fuel consumption
and waste generation. Many reactors of this generation of differing designs are in operation
and construction around the globe \footcite[7-11]{Gen2gen}.