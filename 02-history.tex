\chapter{Historical Generational Developments of Nuclear Reactors}
Nuclear reactors are normally categorised into generations. These generations differentiate
reactors based on various technological factors such as cost-effectiveness, safety, commercial applicability and
fuel cycle. These generations present a useful tool in categorizing nuclear reactors as multiple factors
are combined into one single metric. It is important to not  that all designs of generations I
to III+ consist mainly of designs first developed in the late fourth decade of the 20th century \footcite[1,2]{Gen2gen}.
\section{Generation I}
The first generation of nuclear reactors consists primarily of research reactors and primitive nuclear
power plants. These reactors are regarded as \enquote{proof of concept} in the USA. All reactors of
this generation have been decommissioned or are undergoing deconstruction as their technological level
is far behind that of newer reactors. Therefore they have very low cost-effectiveness and operational
safety \footcite[3]{Gen2gen}.
\section{Generation II}
The second generation of nuclear reactors represent the first efforts to produce nuclear reactors primarily
designed for commercial viability. The comprise mainly boiling water reactors (BWR), pressurised water reactors (PWR),
Canadian deuterium uranium reactors (CANDU) and advanced gas-cooled reactors (AGR).  They are designed
for an operational lifetime of 40 years. Generation II reactors are the most common generation for boiling
water reactors and pressurised water reactors around the world, which can be categorised under the term
light water reactor (LWR). These reactors feature more advanced safety features compared to generation I
reactors. These safety have the ability to automatically prevent grievous incidents in the operation
of a nuclear reactors, as they prevent dangerous incidents in cases such as loss of power or operator
error.

Designs from this category require comparatively large electrical power grids. And feature advanced
safety envelopes based on western standards. New reactors of this generation are mainly built in
China, Russia and the Republic of Korea. Their fuel cycle requires high level waste repositories
for ultimate disposition \footcite[4-6]{Gen2gen}.
\section{Generation III}
\section{Generation III+}