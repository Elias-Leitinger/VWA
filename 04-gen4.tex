\chapter{Generation IV}
Nuclear reactor systems of the fourth generation are a set of six novel reactor designs created with
set goals in mind. This development effort is lead and governed by the Generation IV International Forum
(GIF). The GIF has 14 Members: Argentina, Australia, Brazil, Canada, China, Euratom, France, Japan,
Korea, the Russian Federation, South Afrika, Switzerland, the United Kingdom and the United States of Amerika.
The GIF goal is to motivate research and development on new reactor systems\footcite[6]{GIFAR}.
\section{Goals}
Goals have been defined for nuclear reactor systems of the fourth generation. These goals
can be categorised into four sections: Economics, Sustainability, Safety and Reliability and, Proliferation Resistance\footcite[38]{IVHandbook}.
\subsection{Economics}
Generation IV nuclear reactors will have a lower total life-cycle cost when compared to nuclear reactors
of previous generations. This means that over the energy produced by these nuclear reactor systems over
their lifetime is higher in relation to the cost involved in building such a power plant.
Additionally, these reactor systems will have comparable financial risks to other energy projects, which
is especially important to make Generation IV systems competitive in the marketplace. This also includes
viability for a wider range of different ownership models and a wider array of possible energy supply roles\footcite[6]{GIFAR}.
\subsection{Sustainability}
It is important for nuclear reactor systems of the fourth generation to make effective use of the resources
involved in building and operating such a system. Another area of importance is the reduction and effective
management of nuclear waste\footcite[38]{IVHandbook}.
\subsection{Safety and Reliability}
In nuclear reactors of previous generations a compromise between safety and reliability needs to be met,
because these designs have not been designed with safety primarily in mind. Therefore an increase in safety
mostly coincides with reduced reliability as these measures increase the complexity of the nuclear reactor and
also decrease the amount of valid operating states. Because Generation IV nuclear reactors are designed from the ground up
with inherent safety systems no compromise needs to be struck. Most of these novel designs feature passive
protection against failure modes such as overheating\footcite[6]{GIFAR}.
\subsection{Proliferation Resistance and Physical Protection}
Proliferation resistance and physical protection are important to make the deployment of nuclear reactors
attractive. Generation IV reactors are designed to make them the most unattractive option for creating
fissile material for use in nuclear weapons. Therefore their fuel cycle mus produce as few actinides
as possible, which is also beneficial for the sustainability of such systems\footcite[39]{IVHandbook}.
\section{Reactor Types}
\subsection{Very High Temperature Reactor}
Because usually used in nuclear reactors such as zircalloy or steel corrode or melt at temperatures
above 650°C, these materials cannot be used to build reactors, which allow for higher temperatures.
Higher reactor temperatures are of advantage because they increase efficiency significantly. The
only materials such as ceramics can be used for such a purpose\footcite[55]{T4Gen}.
\subsubsection{Fuel Elements}
To allow for increased temperatures, the fuel elements inside a very high temperature reactor are made
from ceramic uranium dioxide spheres and coated with graphite and silicon carbide ($\ce{SiC}$). These
particles have a diameter of 0.9mm. These gas tight particles are called TRISO (TRistructural-ISOtropic) particles.
The particles are then arranged into cylindrical graphite
pellets with a diameter of 24mm or spheres with a diameter of 60mm. The pellets are usually used in
graphite block type reactor cores, which closely resemble those of traditional reactor cores such
as RBMK reactors. However, the fuel spheres are used in pebble bed type reactors, which will be explained
in a following section\footcite[56]{T4Gen}.
\subsubsection{Very High Temperature Reactors with Prismatic Core}
\myfig{VHTR}
    {height=0.4\textheight}
    {Schematic representation of a very high temperature reactor}
    {Schematic representation of a VHTR, public domain}
    {fig:vhtr}
An example of a reactor, which uses the graphite block type core configuration is shown in figure~\ref{fig:vhtr}.
This type of reactor is moderated by graphite and cooled with helium. As a coolant, helium has
a lower thermal capacity compared to water, but because helium can be used at higher operating
temperatures, it's use is advantageous for high core temperatures. Additionally helium has close
to no ablility to moderate neutrons. In such a reactor configuration
the prism shaped core is surrounded by a graphite reflector and a pressure vessel. The primary turbine
is located in a secondary pressure vessel and directly driven by the helium, which is heated by the
reactor core. Heat exchangers are used to transfer the remaining heat from turbines into a water based
cooling loop for additional energy extraction\footcite{VHTRTS}.
\subsubsection{Pebble Bed Reactors}
\myfig{PBR}
    {height=0.4\textheight}
    {Schematic representation of a pebble bed reactor}
    {Schematic representation of a pebble bed reactor, public domain}
    {fig:pbr}
The reactor core of a pebble bed reactor is made up of around 500000 individual fuel spheres. These
spheres rest loosely inside the graphite reflector which is contained by a pressure vessel. The graphite
reflector contains the control rods, because in previous research efforts the control rods were blocked
by jammed spheres. New fuel spheres can be continuously fed in from the top side of the reactor, while
spent fuel is extracted from the bottom end funnel. Therefore the reactor can be operated and refueled
without the shutdowns normally required in refuelling. Helium is blow through the reactor core by a
pump. A heat exchanger is used to extract the generated heat from the nuclear reactor\footcite[60-62]{T4Gen}.
Such a reactor core is illustrated in~\ref{fig:pbr}.
\subsubsection{Safety}
A great advantage of such a reactor is the inherent safety it provides. This means that the core
need not have a emergency cooling system, because the fuel pebbles can survive heat in excess of 1600°C
meaning that a core meltdown is virtually impossible. If the helium cooling fails the core only needs to
be cooled from the outside to dissipate the heat created. Another advantage of this reactor type is, that
no direct contact between the coolant and fissile material is possible due to the coating of the fuel
spheres. Therefore in the event of coolant loss no radioactive material can be leaked
into the surrounding environment\footcite[18-21]{VHTRTS}.
Because they utilize fuel, which is only enriched to 8\%, they have the same proliferation resistance
as the pressurized water reactors currently in use\footcite{VHTRTS}.
\subsubsection{Efficiency}
Because very high temperature reactors operate at elevated temperatures and may make use of turbines
operated directly by the coolant gas, they have comparatively high thermal efficiencies of 42\%.
They, however, are expected to reach efficiencies in excess of 50\% with further development\footcite[62]{T4Gen}.
\pagebreak
\textit{Put others in between}
\pagebreak
\subsection{Molten Salt Reactors}
\myfig{MSR}
    {height=0.4\textheight}
    {Schematic representation of a molten salt reactor}
    {Schematic representation of a molten salt reactor, ©Terrestrial Energy; Creative Commons}
    {fig:msr}
Molten salt reactors are a class of reactors where the fuel is not present inside the core as a solid
but rather as a liquid. In such a configuration the salt used as fuel also functions as the primary coolant
simultaneously. Because the melting points of the salts used are 400°C to 500°C the core operates at
ranging from 650°C to 700°C. However the pressure inside the reactor and coolant systems does never
exceed 1atm meaning that the pressure inside the reactor is never above atmospheric pressure.
If operation with thermal neutrons is desired, graphite is used as the moderator\footcite[147-152]{T4Gen}.
An illustrativ schematic of such a reactor is shown in~\ref{fig:msr}.
\subsubsection{Fuel}
Uranium dioxide is not suitable as a fuel for molten salt reactors, because it has a melting point around
1000°C. Therefore a mixture of different fluoride containing salts is used. The main salts used are
Lithium Fluoride (\ce{LiF}) and Beryllium Fluoride (\ce{BeF_{2}}). The fissile material is present in the
form of Uranium Fluoride (\ce{UF_4}). If the reactor is operated using thermal neutrons, unenriched
uranium may be used as fission fuel.
\subsubsection{Safety}
Because the reactor core is already in the liquid phase during normal operation, a core meltdown is not
possible. Additional, if an emergency condition is detected, the contents of the core can be dumped
into special containers where the reactive mass becomes subcritical and is allowed to cool off.
It is also possible to change the fuel contents during regular core operation, thereby making a core
shutdown unnecessary. This also allows for more streamlined fuel cycles, as the components composing
the molten salt fuel can be separated directly in a secondary plant, thereby bypassing the need for
separate reprocessing facilities and creating the possibility to remove fission products and simultaneously
introduce new fission fuel during reactor operation. Lastly, because gaseous substances such as Iodine
and Xenon Isotopes are not well soluble in liquid salt when compared to water. Therefore
it is easier to extract and capture them\footcite{msrs}. If the reactor does not make use
of a moderator and instead uses fast neutrons, it has the advantage of a negative void effect. This
means that as the heat inside the reactor rises, the reactivity of the reactor decreases, thereby
creating a passive safety feature. Fast molten salt reactors are the only fast reactors with such a
property\footcite[164]{T4Gen}.
\subsubsection{Efficiency}
Because of the higher operating temperature molten salt reactors are more efficient at generating
electrical power. Additionally, because the fuel composition can be varied considerably during operation
and because unwanted isotopes can be extracted or eliminated in a targeted fashion, they may also
profit from high economic viability\footcite[159-161]{IVHandbook}.
\subsubsection{Disadvanages}
Because the liquid fuel comes into contact with a greater deal of moving mechanisms such as pumps, a
greater area of containment and more remote operation and servicing is needed to operate such a reactor
\footcite{msrs}.