\chapter{Generation IV}
Nuclear reactor systems of the fourth generation are a set of six novel reactor designs created with
set goals in mind. This development effort is lead and governed by the Generation IV International Forum
(GIF). The GIF has 14 Members: Argentina, Australia, Brazil, Canada, China, Euratom, France, Japan,
Korea, the Russian Federation, South Afrika, Switzerland, the United Kingdom and the United States of Amerika.
The GIF goal is to motivate research and development on new reactor systems\footcite[6]{GIFAR}.
\section{Goals}
Goals have been defined for nuclear reactor systems of the fourth generation. These goals
can be categorised into four sections: Economics, Sustainability, Safety and Reliability and, Proliferation Resistance\footcite[38]{IVHandbook}.
\subsection{Economics}
Generation IV nuclear reactors will have a lower total life-cycle cost when compared to nuclear reactors
of previous generations. This means that over the energy produced by these nuclear reactor systems over
their lifetime is higher in relation to the cost involved in building such a power plant.
Additionally, these reactor systems will have comparable financial risks to other energy projects, which
is especially important to make Generation IV systems competitive in the marketplace. This also includes
viability for a wider range of different ownership models and a wider array of possible energy supply roles\footcite[6]{GIFAR}.
\subsection{Sustainability}
It is important for nuclear reactor systems of the fourth generation to make effective use of the resources
involved in building and operating such a system. Another area of importance is the reduction and effective
management of nuclear waste\footcite[38]{IVHandbook}.
\subsection{Safety and Reliability}
In nuclear reactors of previous generations a compromise between safety and reliability needs to be met,
because these designs have not been designed with safety primarily in mind. Therefore an increase in safety
mostly coincides with reduced reliability as these measures increase the complexity of the nuclear reactor and
also decrease the amount of valid operating states. Because Generation IV nuclear reactors are designed from the ground up
with inherent safety systems no compromise needs to be struck. Most of these novel designs feature passive
protection against failure modes such as overheating\footcite[6]{GIFAR}.
\subsection{Proliferation Resistance and Physical Protection}
Proliferation resistance and physical protection are important to make the deployment of nuclear reactors
attractive. Generation IV reactors are designed to make them the most unattractive option for creating
fissile material for use in nuclear weapons. Therefore their fuel cycle mus produce as few actinides
as possible, which is also beneficial for the sustainability of such systems\footcite[39]{IVHandbook}.
\section{Reactor Types}
