\chapter{Generation IV}
Nuclear reactor systems of the fourth generation are a set of six novel reactor designs created with
set goals in mind. This development effort is lead and governed by the Generation IV International Forum
(GIF). The GIF has 14 Members: Argentina, Australia, Brazil, Canada, China, Euratom, France, Japan,
Korea, the Russian Federation, South Afrika, Switzerland, the United Kingdom and the United States of Amerika.
The goal of the GIF is to motivate research and development on new reactor systems\footcite[6]{GIFAR}.
\section{Goals}
Goals have been defined for nuclear reactor systems of the fourth generation. These goals
can be categorised into four sections: Economics, Sustainability, Safety and Reliability and, Proliferation Resistance\footcite[38]{IVHandbook}.
\subsection{Economics}
Generation IV nuclear reactors will have a lower total life-cycle cost when compared to nuclear reactors
of previous generations. This means that the energy produced by these nuclear reactor systems over
their lifetime is higher in relation to the cost involved in building and operating such a power plant.
Additionally, these reactor systems will have comparable financial risks to other energy projects, which
is especially important to make Generation IV systems competitive in the marketplace. This also includes
viability for a wider range of different ownership models and a wider array of possible energy supply roles\footcite[6]{GIFAR}.
\subsection{Sustainability}
It is important for nuclear reactor systems of the fourth generation to make effective use of the resources
involved in building and operating such a system. Another area of importance is the reduction and effective
management of nuclear waste\footcite[38]{IVHandbook}.
\subsection{Safety and Reliability}
In nuclear reactors of previous generations a compromise between safety and reliability needs to be met,
because these designs have not been created with safety primarily in mind. Therefore an increase in safety
mostly coincides with reduced reliability as these measures increase the complexity of the nuclear reactor and
also decrease the amount of valid operating states. Because Generation IV nuclear reactors are designed from the ground up
with inherent safety systems, no compromise needs to be struck. Most of these novel designs feature passive
protection against failure modes such as overheating\footcite[6]{GIFAR}.
\subsection{Proliferation Resistance and Physical Protection}
Proliferation resistance and physical protection are important to make the deployment of nuclear reactors
attractive. Generation IV reactors are designed to make them the most unattractive option for creating
fissile material for use in nuclear weapons. Therefore their fuel cycle must produce as few actinides
as possible, which is also beneficial for the sustainability of such systems\footcite[39]{IVHandbook}.
\section{Reactor Types}
\subsection{Supercritical Water Reactors}
\myfig{SWR}
    {height=0.4\textheight}
    {Schematic representation of a supercritical water reactor}
    {Schematic representation of a supercritical water reactor, public domain}
    {fig:swr}
\myfig{phase}
    {height=0.4\textheight}
    {Phase change diagram of \ce{H_{2}O}}
    {Phase change diagram of \ce{H_{2}O}, GNU free documentation license}
    {fig:phase}
Construction wise, supercritical water reactors are very similar to boiling water reactors. They
use light water for the cooling and moderation of the reactor. The main
difference is the aggregate state change of the water used as the coolant. In a boiling water reactor, the
water changes aggregate state at the set boiling point. In a supercritical water reactor, the coolant
is heated beyond the critical temperature of 374°C. Therefore, there is no difference in the density
of liquid particles and steam. This effectively creates a homogenous mass of \ce{H_{2}O}, where no clear
distinction of steam and liquid water can be made. This state is called supercritical and should
not be confused with the term of criticality used in nuclear physics\footcite[206-236]{IVHandbook}. The effect of supercriticality
is well illustrated by the phase change diagram of water shown in figure~\ref{fig:phase}.
A schematic representation of a supercritical boiling water reactor is shown in figure~\ref{fig:srw}.
\subsubsection{Efficiency}
The main advantage of supercritical water reactor lay in the increased efficiency compared to boiling
water reactors because the turbines are operated at higher temperature and pressure and because
the turbine exhaust is fed into the reactor without the need to dump thermal energy to liquefy it again.
This is expected to increase the thermal efficiency from $\approx35\%$ to $\approx45\%$. Such a system
is already used for the energy generation of many coal power plants built since 1990\footcite[30-51]{T4Gen}.
\subsubsection{Safety}
Because of the great similarity to boiling water reactors, supercritical water reactors are comparable
in safety to the aforementioned. However, because no reactor of this type has been built to date,
little research has been conducted on the practical safety of those systems.
Additionally, due to the temperature and density gradient in the coolant inside the core itself,
the moderating coolant absorbs fewer neutrons in the top section of the reactor. Therefore,
an additional moderator needs to be installed at points of higher coolant temperature\footcite[52-53]{T4Gen}.
\subsubsection{Disadvanages}
Supercritical water reactors have the same unsustainable fuel cycle of boiling water reactors,
because they operate on enriched uranium fuel and generate nuclear waste. Therefore they are
not considered appropriate for wide scale adoption. This is also reflected in the small amount
of research efforts that have gone into this system.
An additional disadvantage of such a system is the need for the coolant to be superheated for
safe operation and the high operation pressure which is exerted on the reactor vessel\footcite[52-53]{T4Gen}.
\pagebreak
\subsection{Very High Temperature Reactors}
Very high temperature reactors operate at temperatures far exceeding normal reactor operating limits.
Higher reactor temperatures are of advantage because they increase efficiency significantly.
Because materials usually used in fuel elements such as zircalloy or steel corrode or melt at temperatures
above 650°C, these materials cannot be used for their construction.
The only materials such as ceramics can be used for such a purpose\footcite[55]{T4Gen}.
\subsubsection{Fuel Elements}
To allow for increased temperatures, the fuel elements inside a very high temperature reactor are made
from ceramic uranium dioxide spheres and coated with graphite and silicon carbide ($\ce{SiC}$). These
particles have a diameter of \unit[0.9]{mm}. These gas tight particles are called TRISO (TRistructural-ISOtropic) particles.
The particles are then arranged into cylindrical graphite
pellets with a diameter of 24mm or spheres with a diameter of 60mm. The pellets are usually used in
graphite block type reactor cores, which closely resemble those of traditional reactor cores such
as RBMK reactors\footcite[56]{T4Gen}. However, the fuel spheres are used in pebble bed type reactors, which will be explained
in section~\ref{chap:pbr}.
\subsubsection{Very High Temperature Reactors with Prismatic Core}
\myfig{VHTR}
    {height=0.4\textheight}
    {Schematic representation of a very high temperature reactor}
    {Schematic representation of a very high temperature reactor, public domain}
    {fig:vhtr}
An example of a reactor, which uses the graphite block type core configuration is shown in figure~\ref{fig:vhtr}.
This type of reactor is moderated by graphite and cooled with helium. As a coolant, helium has
a lower thermal capacity compared to water, but because helium can be used at higher operating
temperatures, it's use is advantageous for high core temperatures. Additionally helium has close
to no ablility to moderate neutrons. In such a reactor configuration
the prism shaped core is surrounded by a graphite reflector and a pressure vessel. The primary turbine
is located in a secondary pressure vessel and directly driven by the helium, which is heated by the
reactor core. Heat exchangers are used to transfer the remaining heat from turbines into a water based
cooling loop for additional energy extraction\footcite{VHTRTS}.
\subsubsection{Pebble Bed Reactors}\label{chap:pbr}
\myfig{PBR}
    {height=0.4\textheight}
    {Schematic representation of a pebble bed reactor}
    {Schematic representation of a pebble bed reactor, public domain}
    {fig:pbr}
The reactor core of a pebble bed reactor is made up of around 500000 individual fuel spheres. These
spheres rest loosely inside the graphite reflector which is contained by a pressure vessel. The graphite
reflector contains the control rods, because in previous research efforts the control rods were blocked
by jammed spheres. New fuel spheres can be continuously fed in from the top side of the reactor, while
spent fuel is extracted from the bottom end funnel. Therefore the reactor can be operated and refueled
without the shutdowns normally required. Helium is blow through the reactor core by a
pump. A heat exchanger is used to extract the generated heat from the nuclear reactor\footcite[60-62]{T4Gen}.
Such a reactor core is illustrated in~\ref{fig:pbr}.
\subsubsection{Safety}
A great advantage of such a reactor is the inherent safety it provides. This means that the core
need not have a emergency cooling system, because the fuel pebbles can survive heat in excess of 1600°C
meaning that a core meltdown is virtually impossible. If the helium cooling fails, the core only needs to
be cooled from the outside to dissipate the heat created. Another advantage of this reactor type is, that
no direct contact between the coolant and fissile material is possible due to the coating of the fuel
spheres. Therefore, in the event of coolant loss no radioactive material can be leaked
into the surrounding environment\footcite[18-21]{VHTRTS}.
Because they utilize fuel, which is only enriched to 8\%, they have the same proliferation resistance
as the pressurized water reactors currently in use\footcite{VHTRTS}.
\subsubsection{Efficiency}
Because very high temperature reactors operate at elevated temperatures and may make use of turbines
operated directly by the coolant gas, they have comparatively high thermal efficiencies of 42\%.
They, however, are expected to reach efficiencies in excess of 50\% with further development\footcite[62]{T4Gen}.
\pagebreak
\subsection{Sodium-Cooled Fast Reactors}\label{chap:sfr}
\myfig{SFR}
    {height=0.4\textheight}
    {Schematic representation of sodium-cooled fast reactor}
    {Schematic representation of sodium-cooled fast, public domain}
    {fig:sfr}
Sodium cooled fast reactors already have a working history of 60 years. Like all fast reactors, they make
use of fast neutrons with energies exceeding 3keV. Because fast neutrons have a nuclear cross section which is lower than that
of thermal neutrons by 500\%, a neutron flux ten times greater than that of thermal neutron based reactors
is required to achieve a comparable energy density\footcite[120-122]{ReactorPhysics}.
Sodium-cooled fast reactors utilize sodium as the primary and secondary coolant. Sodium has little
capability to interact with neutons. It is therefore well suited for fast reactors. Compared
to water, sodium has a thermal conductivity greater than that of water by a factor of 100. Because
it has a melting point of 98°C and a boiling point of 883°C it is possible to operate the reactor at an
entry temperature of 400°C and an exit temperature of 550°C while keeping the pressure at 1atm. Sodium
as also very compatible with stainless steel, leading to no erosion of the reactor hull. The core
itself is made up of conventional fuel rods and submerged in a circulating pool of liquid sodium.
To increase the energy of the neutrons the fuel rods are mounted closer together to minimise the
already small moderating effect of the coolant\footcite[94-110]{T4Gen}. Figure~\ref{fig:sfr} illustrates
such a reactor.
\subsubsection{Fuel}
Because of the high energy neutrons, a large amount of plutonium in the form
plutonium dioxide \ce{PuO_{2}} is added to the fuel. The mixed oxide fuel is made up of 15\% \ce{Pu_{2}}
with the rest being \ce{UO_{2}}. During the reactor operation more plutonium is created as a fission product.
Most notably, more fissile fuel is created from natural uranium than is consumed in the process.
Additionally the actinides created by the reaction are recycled in further fission reactions leading
to a closed fuel cycle. No transuranic waste is created by fast reactors. The primary
factor limiting the lifetime of fuel elements is the damage caused to the fuel containing tubes
caused by the high neutron flux\footcite[111]{T4Gen}.
\subsubsection{Safety}
Because of the low operation pressure the risk of containment failure is kept at a minimum. Additionally,
the chance of environmental pollution is reduced because there remains little to no radioactive transuranic
waste to be stored away. The primary operating risk is the high reactivity of sodium, especially with water.
Therefore even contact with miniscule amounts of \ce{H_{2}O}, such as those in the atmosphere, could
lead to catastrophic chemical explosions. Great care needs to be taken to protect the cooling
sodium from water. Additionally, caution needs to be exercised to keep the sodium coolant
below its boiling point of 883°C to avoid the creation of a positive void coefficient, which
would create a fatal super critical state. Lastly, reactors of this type have a self regulating
property, as the core expands with increased heat. This allows more neutrons to escape the
core, reducing reactivity\footcite[30-37]{GIFAR}.
\subsubsection{Efficiency}
Sodium-cooled fast reactors can reach high thermal efficiencies comparable to those of other fast
reactors. Also, to reduce mechanical failures, electrodynamic pumps can be used to propel the
molten sodium instead of mechanical pumps, reducing the amount of moving parts and thereby increasing
reliability\footcite[30-37]{GIFAR}. 
\pagebreak
\subsection{Lead-Cooled Fast Reactors}
\myfig{LFR}
    {height=0.4\textheight}
    {Schematic representation of sodium-cooled fast reactor}
    {Schematic representation of sodium-cooled fast, public domain}
    {fig:lfr}
The construction of lead cooled fast reactors is very similar to that of sodium-cooled fast reactors,
as can be seen in figure~\ref{fig:lfr}. For the sake of brevity, only the differences to sodium-cooled
fast reactors will be illustrated, while the parallels are omitted.\\
The two main problems of using sodium as a coolant can be solved by replacing it with another metal.
Lead is well suited for such an application due to its relatively high melting point of 337°C and
the high boiling temperature of 1750°C. Therefore it is impossible for the lead to start boiling inside
the reactor, because the fuel assemblies and the reactor hull would be destroyed first.
Another advantage is the ability of lead to naturally circulate through the core because of the differences
in density of lead at different temperatures.
Because of the relatively high melting temperature great caution needs to be exercised as to not freeze the coolant
inside the reactor. To mitigate this possibility the alloy lead-bismuth may be used. Lead-bismuth consists
of 55.5 mass \% lead and 44.5 mass \% bismuth. This alloy has a low melting temperature 124°C, but also
a decreased boiling point of 1670°C. Except for the higher nuclear cross section of lead-bismuth, the
two coolants can be used interchangeably. To further examine lead and bismuth as coolants close attention
needs to be drawn towards the individual isotopes of these metals\footcite[137-174]{IVHandbook}.
\subsubsection{Coolant Isotopes}
As can be seen from table~\ref{lead}, \ce{^{204}_{82}Pb} has a higher nuclear cross section of neutron
absorption compared to bismuth, but because naturally occurring lead
consists of a mixture of the lead isotopes listed in~\ref{lead} in the ratios given, its total
cross section of neutron absorption is lower compared to bismuth. Through neuton capture~\ce{^{206}_{82}Pb} and
~\ce{^{207}_{82}Pb} are converted to \ce{^{208}_{82}Pb} during the nuclear reactor operation. Through
further neutron absorption \ce{^{209}_{82}Pb} can be created, which decays to \ce{^{209}_{83}Bi}
with a half life of 3.23h. When \ce{^{209}_{83}Bi} absorbs a neutron, \ce{^{210}_{83}Bi} which
decays to \ce{^{210}_{84}Po} with a half life of 5 days, a strong alpha emitter with a half life
of 124 days. \ce{^{210}_{84}Po} decays to \ce{^{204}_{82}Pb}. Thereby the coolant is not only contaminated,
but there is a significant safety risk for the operating personell, but due to the short half life
of the relevant isotopes, one only has to await the decay of the isotopes if disposal is necessary\footcite[115-134]{T4Gen}.
\begin{table}[h!]\label{lead}
    \begin{tabular}{llllll}
    Isotope                                     & \ce{^{204}_{82}Pb} & \ce{^{206}_{82}Pb} & \ce{^{207}_{82}Pb} & \ce{^{208}_{82}Pb} & \ce{^{209}_{83}Bi} \\
    Abundance (\%)                              &   1.4              &       24.1         &       22.1         &         52.4       &  100  \\
    Half-Life (years)                           &   $1.4*10^{17}$    &     Stable         &     Stable         &     Stable         &   $2*10^{19}$ \\
    Nuclear cross section (mbarn) &         703        &        26.6        &      622           &        0.23        &   32.4 \\ \hline
    \multicolumn{6}{l}{Table~\ref{lead}}
    \end{tabular}
    \end{table}
\subsubsection{Safety}
Because lead is not susceptible to explosive reactions with water and because it mitigates the problem
of a positive void coefficient it is better suited a as coolant for a fast rector. Additionally,
in the case of a breach in the reactor vessel, the leaking lead solidifies below melting temperature
and thereby stops further evacuation of the reactor chamber. The main problem of safety in operation
is the danger of irradiation caused by the contamination of the lead coolant through the generation
of radioactive isotopes, mainly \ce{^{210}_{84}Po}\footcite[132]{T4Gen}.
\subsubsection{Efficiency}
The efficiency and fuel cycle benefits are identical to those of sodium-cooled fast reactors. See chapter~\ref{chap:sfr}.
\pagebreak
\subsection{Gas-Cooled Fast Reactors}
\myfig{GFR}
    {height=0.4\textheight}
    {Schematic representation of a gas-cooled fast reactor}
    {Schematic representation of a gas-cooled fast reactor, public domain}
    {fig:gfr}
Gas-cooled fast reactors employ a gas, in all relevant cases helium, as the coolant. Thermodynamically,
they bear a great similarity to helium-cooled very high temperature reactors. But, because they
are fast reactors, the core design varies greatly. The core assembly is very similar to sodium-cooled
fast reactors, where mixed oxide fuel is housed in stainless steel tubes. Attempts to use the more
economical and safe TRISO pellets have failed because of damages caused by high neutron flux. Because
of its very low density, helium has no ability to moderate neutrons with a cross section of neutron
absorption of 0 mbarn. This makes it the ideal coolant for a fast reactor. However, if operated at
70 atm of pressure, the flow rate of the reactor coolant needs to 500 times as high compared to a
sodium-cooled fast reactor to achieve comparable heat transfer rates\footcite[135-144]{T4Gen}.
\subsubsection{Safety}
Because of the high flow rates required to keep the reactor operational, active pumps need to be employed
to move the coolant helium through the core. Because of the high pressure it is very difficult to design
a passive safety system for such a reactor and no tested solution has been found yet, because no such
system has ever reached criticality. However, because helium is chemically inert, no corrosion of the
reactor vessel can be expected. Additionally, no positive void coefficient can occur because helium
already has no ability to moderate neutrons\footcite[135-144]{T4Gen}.
\subsubsection{Efficiency}
The efficiency and fuel cycle benefits are identical to those of sodium-cooled fast reactors. See chapter~\ref{chap:sfr}
\pagebreak
\subsection{Molten Salt Reactors}
\myfig{MSR}
    {height=0.4\textheight}
    {Schematic representation of a molten salt reactor}
    {Schematic representation of a molten salt reactor, ©Terrestrial Energy; Creative Commons}
    {fig:msr}
Molten salt reactors are a class of reactors where the fuel is not present inside the core as a solid
but rather as a liquid. In such a configuration the salt used as fuel also functions as the primary coolant
simultaneously. Because the melting points of the salts used are 400°C to 500°C the core operates at temperatures
ranging from 650°C to 700°C. However the pressure inside the reactor and coolant systems does never
exceed 1atm meaning that the pressure inside the reactor is never above atmospheric pressure.
If operation with thermal neutrons is desired, graphite is used as the moderator\footcite[147-152]{T4Gen}.
An illustrativ schematic of such a reactor is shown in~\ref{fig:msr}.
\subsubsection{Fuel}
Uranium dioxide is not suitable as a fuel for molten salt reactors because it has a melting point around
1000°C. Therefore a mixture of different fluoride containing salts is used. The main salts used are
Lithium Fluoride (\ce{LiF}) and Beryllium Fluoride (\ce{BeF_{2}}). The fissile material is present in the
form of Uranium Fluoride (\ce{UF_4}). If the reactor is operated using thermal neutrons, unenriched
uranium may be used as fission fuel.
\subsubsection{Safety}
Because the reactor core is already in the liquid phase during normal operation, a core meltdown is not
possible. Additionally, if an emergency condition is detected, the contents of the core can be dumped
into special containers where the reactive mass becomes subcritical and is allowed to cool off.
It is also possible to change the fuel contents during regular core operation, thereby making a core
shutdown unnecessary. This also allows for more streamlined fuel cycles, as the components composing
the molten salt fuel can be separated directly in a secondary plant, thereby bypassing the need for
separate reprocessing facilities and creating the possibility to remove fission products and simultaneously
introduce new fission fuel during reactor operation. Lastly, gaseous substances such as Iodine
and Xenon Isotopes are not well soluble in liquid salt when compared to water. Therefore,
it is easier to extract and capture them\footcite{msrs}. If the reactor does not make use
of a moderator and instead uses fast neutrons, it has the advantage of a negative void effect. This
means that as the heat inside the reactor rises, the reactivity of the reactor decreases, thereby
creating a passive safety feature. Fast molten salt reactors are the only fast reactors with such a
property\footcite[164]{T4Gen}.
\subsubsection{Efficiency}
Because of the higher operating temperature, molten salt reactors are more efficient at generating
electrical power. Additionally, because the fuel composition can be varied considerably during operation
and because unwanted isotopes can be extracted or eliminated in a targeted fashion, they may also
profit from high economic viability\footcite[159-161]{IVHandbook}.
\subsubsection{Disadvanages}
Because the liquid fuel comes into contact with a greater deal of moving mechanisms such as pumps, a
greater area of containment and more remote operation and servicing is needed to operate such a reactor
\footcite{msrs}.