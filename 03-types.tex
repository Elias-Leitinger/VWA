\chapter{Common Nuclear Reactor Types in Current Use} 
\section{PWR – Pressurized Water Reactor}
\myfig{PWR}
    {height=0.4\textheight}
    {Schematic representation of a pressurized water reactor}
    {Schematic representation of a pressurized water reactor, rights granted from: https://world-nuclear.org/gallery/reactor-diagrams/pressurized-water-reactor.aspx}
    {fig:pwr}
Pressurized water reactors harness $\ce{H_{2}O}$ as coolant and commonly also as moderator. However, they have at least
two coolant circuits. The water in the primary coolant circuit is under a large pressure in order
to remain liquid even at the high temperatures generated inside the reactor core. Between the primary
and secondary coolant circuit lay a steam generator, which turns the cool water supplied in the secondary circuit
into steam used to drive turbines. Special attention needs to be drawn to the fact, that primary and secondary
circuit are never directly connected and therefore no radioactive material can be passed from inside the reactor core
though the steam generator. Therefore only the reactor components which are situated before the steam generator
need to be under containment.~\footcite{WNPR}\textsuperscript{,}\footcite[14-84]{engHandbook} 
A simplified schematic representation of a pressurized water reactor can be seen in~\ref{fig:pwr}. The
turbine and cooling systems of every schematic are not shown, as this would exceed the scope of the
explanations given herein and not contribute to the understanding of the reactor type shown.
\pagebreak
\section{BWR – Boiling Water Reactor}
\myfig{BWR}
    {height=0.4\textheight}
    {Schematic representation of a boiling water reactor}
    {Schematic representation of a boiling water reactor, rights granted from: https://world-nuclear.org/gallery/reactor-diagrams/boiling-water-reactor.aspx}
    {fig:bwr}
Boiling water reactors are characterized by their mode of steam production. In comparison to pressurized
water reactors, these reactors boil the water directly inside the reactor core. This has the advantage
of significantly increasing the simplicity of reactor construction, because there is no need for a steam
generator or secondary coolant loop. Like pressurized water reactors, boiling water reactors most commonly
employ water as their moderator. Because the coolant water unavoidably comes into direct contact with
the fuel rods, radioactive isotopes are leaked into the coolant water and also the turbines. But 
as these isotopes almost unanimously consist of $\ce{^{16}N}$, which has a short half life of $7~$seconds.
the access radiation is almost completely depleted after power generation. But due to the further
spread of radioactive material a greater area of containment needs to be constructed for the nuclear
power plant \footcite[85-140]{engHandbook}. A schematic representation of a boiling water reactor
is shown in~\ref{fig:bwr}
\pagebreak
\section{PHWR – Pressurized Heavy Water Reactor}
Pressurized heavy water reactors also called CANDU for Canadian deuterium uranium reactor, utilize
heavy water ($\ce{^{3}_{1}H_{2}O}$ – also called $\ce{D_{2}O}$) as coolant and moderator. The cylindrical
fuel elements, which consist of multiple small fuel rods welded together, as shown in~\ref{fig:fuelelement},
rest inside zirconium alloy pressure tubes, through which the cooling heavy water flows. Many of these tubes are
contained inside the calandria. The calandria itself is an enclosed chamber, which contains low pressure,
low temperature heavy water used only as moderator. Because only the heavy water inside the zirconium alloy
pressure tubes is under high pressure, the calandria need not be able to withstand the pressures, which
are exerted upon the vessels housing pressurize water reactors and heavy water reactors. The only parts
which need to withstand high mechanical stress are the pressure tubes, which can more easily be mass
manufactured, due to their repetitiveness. Another advantage of this design is that single tubes can
be refueled  during reactor operation, as they can be disconnected, while other reactor designs need to be shut down completely
before a change of fuel rods can occur. Pressurized heavy water reactors can also operate on unenriched
natural uranium \footcite[141-198]{engHandbook}. For a simplified representation of a reactor of this
type see figure~\ref{fig:phwr}.
\myfig{PHWR}
    {height=0.4\textheight}
    {Schematic representation of a pressurized heavy water reactor}
    {Schematic representation of a pressurized heavy water reactor, rights granted from: https://world-nuclear.org/gallery/reactor-diagrams/pressurized-heavy-water-reactor.aspx}
    {fig:phwr}
\myfig{fuelelement}
    {height=0.2\textheight}
    {Fuel element as used in PHWR}
    {Fuel element as used in PHWR, rights granted from: https://cna.ca/reactors-and-smrs/nuclear-fuel/}
    {fig:fuelelement}
\pagebreak