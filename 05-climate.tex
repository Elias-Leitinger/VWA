\chapter{Nuclear Power and Climate Change}
It is agreed upon that climate change presents an issue for human society as a whole. Currently,
carbon emissions produced by energy generation, especially through fossile fuels,
present the biggest problem in reaching the goals set forth in the Paris agreement\footcite{paris}.
Therefore a large consensus has formed, that nuclear energy should at least be taken into consideration
in combating climate change\footcite{nucandclim}.
The lifecycle carbon footprint of nuclear energy is among the lowest compared to other energy
sources on the market. According to estimations around 12 grammes of \ce{CO_{2}} are emitted for
every kilowatt hour of produced electricity. This value lies even below solar energy with \unit[40]{g/kWh}
and coal power with \unit[820]{g/kWh}\footcite{ipcc}. However, it is not possible to scale
nuclear power reactors currently in use to a global scale, because of the need for rare materials
in the construction of such a plant, their inefficient fuel cycle and the difficulty of finding a
suitable location\footcite{scale}. But generation IV nuclear reactors are developed to be more
scalable and easily buildable. It is, however, still uncertain as to what degree even the newest
technological developments can be scaled\footcite{GIFAR}.
\section{Fuel}
Often times the shortage of fissile material is cited the limiting factor for the sustainability of
nuclear power generation through fission. This however, only takes reactors operating on the fission
of \ce{^{235}U} into account. It has been shown that fast neutron reactors can turn fertile isotopes
such as \ce{^{238}U} into fissile fuel at a faster rate than they consume it. This means nuclear
power generation can be sustained on the abundantly available \ce{^{238}U}, which is projected to
be practically inexhaustible\footcite{inex}.