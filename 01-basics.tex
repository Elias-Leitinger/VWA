\chapter{Basic Concepts of Nuclear Power}
Nuclear power reactors harness the heat generated by splitting atoms of certain
elements in a controlled and predictable way. This heat is used to create electrical power\footcite{WNPR}.
\section{Fission}
Nuclear fission is the spontaneous or induced reaction, by which an atom is broken up. In the
case of nuclear power reactors, these reactions are exothermic. Nuclear radiation such as in equation~\ref{decay}
already liberates a large amount of energy.
\begin{equation} \label{decay}
    \ce{^{238}_{92}U} \rightarrow \ce{^{234}_{90}Th} + \alpha, ~~P = 8 \cdot 10^{-9} ~\frac{W}{g}
\end{equation}
This power is increased in nuclear reactors by 10 orders of magnitude. Although the effective lifespan
is lowered from $4.468 \cdot 10^{9}$ years to a few months. Therefore fission is the main reaction
trough which nuclear reactors generate the majority or their power output. An example of such
a reaction is given in equation~\ref{fiss}. \footcite[286]{nucfundamentals}
\begin{equation} \label{fiss}
\begin{split}
    \ce{^{235}_{92}U} + n & \rightarrow \ce{^{236}_{92}U} \\
    \ce{^{236}_{92}U} & \rightarrow \ce{^{144}_{56}Br} + \ce{^{39}_{36}Kr} + 3n + 177 ~MeV
\end{split}
\end{equation}
It is important to note that~\ref{fiss} is a simplification of the actual decay series of $\ce{^{236}_{92}U}$ into
stable end products.~\ref{fiss} is sufficient to understand the principle behind nuclear fission.
A decay series of $\ce{^{236}_{92}U}$ with no intermediates removed is given in~\ref{fiss2}. \footcite[287]{nucfundamentals}
\begin{equation} \label{fiss2}
    \begin{split}
        \ce{^{236}_{92}U} \rightarrow \ce{^{137}_{53}I} + \ce{^{96}_{39}Y} & + 3n \\
            \ce{^{137}I} \rightarrow \ce{^{137}Xe} ~+~ & e^{-} ~+~ \bar{v}_{e}, ~~t_{1/2}=~24.5s \\
                \ce{^{137}Xe} \rightarrow & \ce{^{137}Cs} ~+~ e^{-} ~+~ \bar{v}_{e}, ~~t_{1/2}=~3.818min \\
                    \ce{^{137}Cs}& \rightarrow \ce{^{137}Ba} ~+~ e^{-} ~+~ \bar{v}_{e}, ~~t_{1/2}=~30.07a \\
            \ce{^{96}Y} \rightarrow \ce{^{96}Zr} ~+~ & e^{-} ~+~ \bar{v}_{e}, ~~t_{1/2}=5.36s 
    \end{split}
\end{equation}

\section{Nuclear Cross Section}
Nuclear cross section describes the probability of a certain nuclear reaction to occur. This
aspect needs thorough consideration in the design of nuclear reactors, as the nuclear cross section
generally increases with the inverse of the velocities of the reactants \footcite[108]{nucfundamentals}.

\section{Criticality}
Criticality is the operating condition of a nuclear reactor, in which the neutrons produced by fission
events is sufficient to sustain a chain reaction. It is measured using the multiplication factor
$k$. It is semantically defined as in~\ref{multfact}. For the exact mathematical definition see 
appendix~\ref{app:sixfact}. If the factor $k$ equals $1$ the reaction is said to be
critical. In a critical reaction, the number of neutrons causing a fission event remains constant. 
If the factor $k$ is less than $1$, the number of neutrons is decreasing and the chain reaction is said to be subcritical. If the factor
$k$ is grater than $1$ the number of neutrons is increasing exponentially and the reaction is supercritical.
\footcite[308]{nucfundamentals}\textsuperscript{,} \footcite[39]{ReactorPhysics}.
\begin{equation}
    \label{multfact}
    k~=~\frac{\textit{number of neutrons involved in fission in generation n}}{\textit{number of neutrons involved in fission in generation n - 1}}
\end{equation}
In practice the reactivity $\rho$ is used. It describes the change of the reactor away from the critical
state. It is calculated using~\ref{react}.
\begin{equation}
    \label{react}
    \rho = \frac{k-1}{k}
\end{equation}

\section{Safety}
Safety is particularly important for nuclear power reactors, as they contain large amounts of radioactive
material, which could be released into the environment in the case of an accident. During the ongoing
fission reaction a large amount of radioactive isotopes, of which actinides are the most dangerous.
Thus safety in nuclear power plants has three main objectives. Firstly, the reactor needs to operate normally
without exposing operators and the environment to dangerous levels of radiation. Secondly, accidents need
to be prevented as much as possible. Thirdly, in the case of an accident the consequences need to be minimized.
Therefore for each reactors risks need to be carefully evaluated and their probabilities need to be
carefully considered~\footcite[793]{engHandbook}.

\section{Efficiency}
Like other power plants, current nuclear power reactors offer efficiencies between 30\% and 35\%.
However with increased heat and cogeneration the efficiency of nuclear power plants could be increased
dramatically~\footcite{energyed}.
\subsection{Cogeneration}
Usually a large amount of energy is lost to the environment in the form of heat. This is usually destructive
towards the environment and results in reduced efficiency. This heat could be used in cogeneration to
supply heating to other facilities or private housing, thereby reducing the amount of wasted, unused energy
~\footcite{iaeaeff}.
\section{Fuel Cycle}
Currently, spent fuel from nuclear reactors is sent to reprocessing facilities. There light actinides are
separated from the remaining uranium and plutonium. Currently these actinides are treated as waste and
stored in facilities to keep them from contaminating the environment. The uranium and plutonium remain
in the form of oxides and are reused for mixed oxide fuels (MOX)\footcite[82-84]{T4Gen}.

\subsection{Proliferation}
A significant risk in the operation of nuclear power reactors is the proliferation of fission products for
use in nuclear weapons. As older fast nuclear reactors produce a great deal of $\ce{^{239}_{94}Pu}$.
Therefore, proliferation resistance is a big consideration in the construction and development of new
power reactors and technologies. A possible mitigation is the early removal of fuel, which is only
partially spent, from the reactor. This fuel contains nearly no $\ce{^{239}_{94}Pu}$, but instead contains
$\ce{^{238}_{94}Pu}$, which cannot be used in weapons and cannot practically be turned into $\ce{^{239}_{94}Pu}$\footcite[87]{T4Gen}.

\section{Components of Nuclear Reactors}
\subsection{Fuel}
The reactor fuel is the fissile material used in the fission reaction inside of a reactor. In
most cases uranium dioxide \ce{UO_{2}} pressed into pellets is used for this purpose. These
pellets are put into tubular fuel rods. The whole fuel assembly inside the reactor consists of many
such rods \footcite{WNPR}.
\subsubsection{Startup Neutron Source}
As the fission of uranium produces on average 2.4 neutrons per reaction, there does not need to be
a constant external influx of neutrons. However, to start this chain reaction inside a new reactor
equipped with newly made fuel rods a neutron source is needed. Usually beryllium combined with an
alpha emitter is used for this purpose, as the collision of an $\alpha$-particle with $\ce{^{9}_{4}Be}$
releases a neutron as part of its reaction, as can be seen in equation~\ref{neut} \footcite{WNPR}\textsuperscript{,}\footcite[100]{nucfundamentals}.

\begin{equation}
    \label{neut}
    \ce{^{9}_{4}Be} ~+~ \ce{^{4}_{2}He} \rightarrow \ce{^{12}_{6}C} ~+~ n
\end{equation}

\pagebreak
\subsection{Moderator}
Nuclear fission events release neutrons with energies in excess of multiple MeV, or speeds higher
than $10^{7} \frac{m}{s}$. However at these speeds the nuclear cross section of the fission reaction
is quite low as can be seen in figure~\ref{fig:ucross}.

\myfig{crosssection}%% filename in figures folder
  {height=0.4\textheight}%% maximum width/height, aspect ratio will be kep
  {Nuclear cross section of $\ce{^{235}U}$ in relation to neutron energy}%% caption
  {Nuclear cross section of $\ce{^{235}U}$ in relation to neutron energy, CC from: https://en.wikipedia.org/wiki/File:U235\_Fission\_cross\_section.png}%% optional (short) caption for list of figures 
  {fig:ucross}%% label

  Therefore the emitted neutrons need to be slowed down in order to be useful. These are called thermal neutrons.
  This is done using a moderator. When passing through a moderator the neutrons are slowed down through
  collisions with the aforementioned. Although it is important to note that the moderator should absorb
  as few neutrons as possible to not hinder the chain reaction. For this reason $\ce{H_{2}O}$ 
  and graphite are commonly used.\footcite[28]{ReactorPhysics}
\subsubsection{Fast Reactors}
However, there is a category of reactors which harness fast neutrons in their nuclear reaction.
These have no moderator and instead make use of fuel, which requires a higher share of $\ce{^{235}U}$
as this reduces the chance of neutron capture by $\ce{^{238}U}$ and increases the likelihood of a fission event to occur.
As actinides are fissile by fast neutrons, fast reactors may reduce the amount of transuranic nuclear waste
generated by nuclear power production. There is still ongoing research regarding fast reactors to make
them useful for widespread energy production.~\footcite{WNPR}


\subsection{Control Rods or Blades}
In order to regulate the reaction speed inside of a nuclear reactor the number of neutrons inducing
nuclear fission needs to be regulated. This is accomplished using control rods.
As a single control rod with a circular cross section would lead to very nonuniform fission and
temperature dynamics, the control rods are either arranged into cruciform blades or evenly spaced
across the reactor in the form of clusters. A typical reactor contains around 50 clusters, each
made up of 20 control rods~\footcite[72]{ReactorPhysics}\textsuperscript{,}\footcite{grayson}.

These rods or blades contain materials which readily absorb neutrons, such as boron or cadmium. %TODO: make more concrete!
They may either be made of steel enriched with boron or hollow tubes filled with a brittle salt like material such as cadmium isotopes.
Because the amount of fuel inside reactor steadily decreases, the amount of neutrons absorbed needs to
be regulated in order for the chain reaction to continue. Therefore the control rods or blades are
mounted on a movable apparatus, which extends or retracts the control rods into or out of the reactor,
thereby regulating the amount of neutrons absorbed\footcite{grayson}.

\subsection{Coolant}
The coolant is a liquid which circulates inside the nuclear reactor core to extract the thermal energy
generated form the fission reactions. In most cases today this liquid is $\ce{H_{2}O}$. In the case
of boiling water reactors the water is boiled directly inside the core. In all other reactor types,
such as pressurized water reactors at least a secondary, separated, coolant circuit is used, which
transports the heat away from the primary circuit. When the water is not boiled inside the reactor
core, a separate steam generator is used to create the steam, which drives the turbine. The same
water which is used as a coolant may also be used as the rector moderator\footcite{WNPR}.
