\chapter{Basic concepts of nuclear power}
Nuclear power reactors harness the heat generated by splitting atoms of certain
elements in a controlled and predictable way. This heat is used to create electrical power\footcite{WNPR}.
\section{Fission}
Nuclear fission is the spontaneous or induced reaction, by which an atom is broken up. In the
case of nuclear power reactors these reactions are exothermic. Nuclear radiation such as~\ref{decay}
already liberates a large amount of energy.
\begin{equation} \label{decay}
    \ce{^{238}_{92}U} \rightarrow \ce{^{234}_{90}Th}, ~~P = 8 \cdot 10^{-9} ~\frac{W}{g}
\end{equation}
This power is increased in nuclear reactors by 10 orders of magnitude. Although the effective lifespan
is lowered from $4.468 \cdot 10^{9}$ years to a few months. Therefore fission is the main reaction
trough which nuclear reactors generate the majority or their power output. An example of such
a reaction is given in~\ref{fiss}. \footcite{nucfundamentals}
\begin{equation} \label{fiss}
\begin{split}
    \ce{^{235}_{92}U} + n & \rightarrow \ce{^{236}_{92}U} \\
    \ce{^{236}_{92}U} & \rightarrow \ce{^{144}_{56}Br} + \ce{^{39}_{36}Kr} + 3n + 177 ~MeV
\end{split}
\end{equation}
It is important to note that~\ref{fiss} is a simplification of the actual decay series of $\ce{^{236}_{92}U}$ into
stable end products.~\ref{fiss} is sufficient to understand the principle behind nuclear fission.
The decay series of $\ce{^{236}_{92}U}$ with no intermediates removed is given in~\ref{fiss2}.
\begin{equation} \label{fiss2}
    \begin{split}
        \ce{^{236}_{92}U} \rightarrow \ce{^{137}_{53}I} + \ce{^{96}_{39}Y} & + 3n \\
            \ce{^{137}I} \rightarrow \ce{^{137}Xe} ~ & e^{-} ~ \bar{v}_{e}, ~~t_{1/2}=24.5s \\
                \ce{^{137}Xe} \rightarrow & \ce{^{137}Cs} ~ e^{-} ~ \bar{v}_{e}, ~~t_{1/2}=3.818m \\
                    \ce{^{137}Cs}& \rightarrow \ce{^{137}Ba} ~ e^{-} ~ \bar{v}_{e}, ~~t_{1/2}=30.07y \\
            \ce{^{96}Y} \rightarrow \ce{^{96}Zr} ~ & e^{-} ~ \bar{v}_{e}, ~~t_{1/2}=5.36s 
    \end{split}
\end{equation}