\chapter{Basic concepts of nuclear power}
Nuclear power reactors harness the heat generated by splitting atoms of certain
elements in a controlled and predictable way. This heat is used to create electrical power\footcite{WNPR}.
\section{Fission}
Nuclear fission is the spontaneous or induced reaction, by which an atom is broken up. In the
case of nuclear power reactors these reactions are exothermic. Nuclear radiation such as in equation~\ref{decay}
already liberates a large amount of energy.
\begin{equation} \label{decay}
    \ce{^{238}_{92}U} \rightarrow \ce{^{234}_{90}Th}, ~~P = 8 \cdot 10^{-9} ~\frac{W}{g}
\end{equation}
This power is increased in nuclear reactors by 10 orders of magnitude. Although the effective lifespan
is lowered from $4.468 \cdot 10^{9}$ years to a few months. Therefore fission is the main reaction
trough which nuclear reactors generate the majority or their power output. An example of such
a reaction is given in equation~\ref{fiss}. \footcite[286]{nucfundamentals}
\begin{equation} \label{fiss}
\begin{split}
    \ce{^{235}_{92}U} + n & \rightarrow \ce{^{236}_{92}U} \\
    \ce{^{236}_{92}U} & \rightarrow \ce{^{144}_{56}Br} + \ce{^{39}_{36}Kr} + 3n + 177 ~MeV
\end{split}
\end{equation}
It is important to note that~\ref{fiss} is a simplification of the actual decay series of $\ce{^{236}_{92}U}$ into
stable end products.~\ref{fiss} is sufficient to understand the principle behind nuclear fission.
The decay series of $\ce{^{236}_{92}U}$ with no intermediates removed is given in~\ref{fiss2}. \footcite[287]{nucfundamentals}
\begin{equation} \label{fiss2}
    \begin{split}
        \ce{^{236}_{92}U} \rightarrow \ce{^{137}_{53}I} + \ce{^{96}_{39}Y} & + 3n \\
            \ce{^{137}I} \rightarrow \ce{^{137}Xe} ~+~ & e^{-} ~+~ \bar{v}_{e}, ~~t_{1/2}=~24.5s \\
                \ce{^{137}Xe} \rightarrow & \ce{^{137}Cs} ~+~ e^{-} ~+~ \bar{v}_{e}, ~~t_{1/2}=~3.818m \\
                    \ce{^{137}Cs}& \rightarrow \ce{^{137}Ba} ~+~ e^{-} ~+~ \bar{v}_{e}, ~~t_{1/2}=~30.07y \\
            \ce{^{96}Y} \rightarrow \ce{^{96}Zr} ~+~ & e^{-} ~+~ \bar{v}_{e}, ~~t_{1/2}=5.36s 
    \end{split}
\end{equation}

\section{Nuclear Cross Section}
Nuclear cross section describes the probability of a certain nuclear reaction to occur. This
aspect needs thorough consideration in the design of nuclear reactors, as the nuclear cross section
generally increases with the inverse of the velocities of the reactants \footcite[108]{nucfundamentals}.

\section{Criticality}
Criticality is the operating condition of a nuclear reactor, in which the neutrons produced by fission
events is sufficient to sustain a chain reaction \footcite[308]{nucfundamentals}\textsuperscript{,} \footcite[39]{ReactorPhysics}.
%TODO: Expand with info from ReactorPhysics

\section{Components of Nuclear Reactors}
\subsection{Fuel}
The reactor fuel is the fissile material used in the fission reaction inside of a reactor. In
most cases uranium oxide \ce{UO_{2}} pressed into pellets is used for this purpose. These
pellets are put into tubular fuel rods. The whole fuel assembly inside the reactor consists of many
such rods \footcite{WNPR}.
\subsubsection{Startup Neutron Source}
As the fission of uranium produces three neutrons per reaction, there does not need to be
a constant external influx of neutrons. However, to start this chain reaction inside a new reactor
equipped with newly made fuel rods a neutron source is needed. Usually beryllium combined with an
alpha emitter is used for this purpose, as the collision of an $\alpha$-particle with $\ce{^{9}_{4}Be}$
releases a neutron as part of its reaction, as can be seen in equation~\ref{neut} \footcite{WNPR}\textsuperscript{,}\footcite[100]{nucfundamentals}.

\begin{equation}
    \label{neut}
    \ce{^{9}_{4}Be} ~+~ \ce{^{4}_{2}He} \rightarrow \ce{^{12}_{6}C} ~+~ n
\end{equation}

\pagebreak
\subsection{Moderator}
Nuclear fission events release neutrons with energies in excess of multiple MeV, or speeds higher
than $10^{7} \frac{m}{s}$. However at these speeds the nuclear cross section of the fission reaction
is quite low as can be seen in figure~\ref{fig:ucross}.

\myfig{crosssection}%% filename in figures folder
  {height=0.4\textheight}%% maximum width/height, aspect ratio will be kep
  {Nuclear cross section of $\ce{^{235}U}$ in relation to neutron energy}%% caption
  {}%% optional (short) caption for list of figures https://en.wikipedia.org/wiki/File:U235_Fission_cross_section.png
  {fig:ucross}%% label

  Therefore the emitted neutrons need to be slowed down in order to be useful, they need to be slowed down.
  This is done using a moderator. When passing through a moderator the neutrons are slowed down through
  collisions with the aforementioned. Although it is important to note that the moderator should absorb
  as few neutrons as possible to not hinder the chain reaction. For this reason $\ce{H_{2}O}$ 
  and graphite are commonly used.\footcite[28]{ReactorPhysics}