\chapter{Basic concepts of nuclear power}
Nuclear power reactors harness the heat generated by splitting atoms of certain
elements in a controlled and predictable way. This heat is used to create electrical power\footcite{WNPR}.
\section{Fission}
Nuclear fission is the spontaneous or induced reaction, by which an atom is broken up. In the
case of nuclear power reactors these reactions are exothermic. Nuclear radiation such as~\ref{decay}
already liberates a large amount of energy.
\begin{equation} \label{decay}
    \ce{^{238}_{92}U} \rightarrow \ce{^{234}_{90}Th}, ~~P = 8 \cdot 10^{-9} ~\frac{W}{g}
\end{equation}
This power is increased in nuclear reactors by 10 orders of magnitude. Although the effective lifespan
is lowered from $4.468 \cdot 10^{9}$ years to a few months. Therefore fission is the main reaction
trough which nuclear reactors generate the majority or their power output. An example of such
a reaction is given in~\ref{fiss}. \footcite{nucfundamentals}
\begin{equation} \label{fiss}
\begin{split}
    \ce{^{235}_{92}U} + n & \rightarrow \ce{^{236}_{92}U} \\
    \ce{^{236}_{92}U} & \rightarrow \ce{^{144}_{56}Br} + \ce{^{39}_{36}Kr} + 3n + 177 ~MeV
\end{split}
\end{equation}